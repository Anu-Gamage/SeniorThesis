\chapter{Review of Related Literature} 

\label{chapter:chapter2}

\section{Multilayer Graph Embedding}
\subsection{Tensor Factorization methods}

\subsection{Spectral Methods}

\subsection{Other methods}

\section{Pros and Cons of these approaches}





%\section{Some background } 


%
%In order to study the trunk of the elephant, my best beloved, we have so many stories to tell you~\cite{kipling2010just}. 
%
%Assume that the trunk consists of $N$ different segments. On the other hand, it is always important to cite the crucial work in the field~\cite{willems2007there}.
%
%%%%%%%%%%%%%%%%%%%%%%%%%%%%%%%%%%%%%%%%
%
%\section{Results} 
%
%
%\subsection{Elephant color scales}
%
%\begin{table}[htb!]
%   \begin{center}
%      \begin{tabular}{|l|l|l|l|l|l|l|} \hline
%      {\bf Class} & {\bf African} & {\bf Asian}  & {\bf Martian}  & {\bf Cartoon}  & {\bf Plush} & {\bf Plastic} \\\hline
%      {\bf Body} &  0.57  & 0.21 & 0.31 & 0.40 & 0.46 & 0.30 \\\hline
%      {\bf Tusk} & 0.85 & 0.87 & 0.87 & 0.91 & 0.93 & 0.93 \\\hline
%      {\bf Trunk} & 1.00  & 0.94 & 0.97 & 0.97 & 0.97 & 1.00 \\\hline
%      {\bf Ear} & 0.73  & 0.73 & 0.64 & 0.73 & 1.00 & 0.82 \\\hline
%      {\bf Tail} & 0.67 & 1.00 & 1.00 & 1.00 & 1.00 & 1.00 \\\hline
%      \end{tabular}
%   \end{center}
%   \caption{Careful measurements in cross-validation. Note that it is easier to get a plush or plastic elephant to stay still and be measured, so those measurements are deemed more reliable. Cartoon elephants require special cleverness and a deep knowledge of cartoon physics to measure with motion capture technology.}
%   \label{gpcr_coverage}
%\end{table}
%
%
%%%%%%%%%%%%%%%%%%%%%%%%%%%%%%%%%%
%
